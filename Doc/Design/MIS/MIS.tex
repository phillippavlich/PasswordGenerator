\documentclass[12pt]{article}
\usepackage[utf8]{inputenc}
\usepackage{tabularx}
\usepackage{booktabs}
\usepackage{pgfgantt}
\usepackage{array}

\newcolumntype{L}[1]{>{\raggedright\let\newline\\\arraybackslash\hspace{0pt}}m{#1}}
\newcolumntype{C}[1]{>{\centering\let\newline\\\arraybackslash\hspace{0pt}}m{#1}}
\newcolumntype{R}[1]{>{\raggedleft\let\newline\\\arraybackslash\hspace{0pt}}m{#1}}
\newcommand{\N}{\mathbb{N}}


\title{3XA3 Module Interface Specification}
\author{Group 20 (2020Vision)\\
Mullen, Thomas - mullentc - 001406837\\
Pavlich, Phillip - pavlicpm - 001414960\\
Bauer, Ivan - bauerim - 001418765\\
L02}
\date{}

\begin{document}
\begin{table}[hp]
\caption{Revision History} \label{TblRevisionHistory}
\begin{tabularx}{\textwidth}{XXl}
\toprule
\textbf{Date} & \textbf{Developer(s)} & \textbf{Change}\\
\midrule
November 10, 2017 & Thomas Mullen & Rev.0 of Document\\
November 10, 2017 & Phillip Pavlich & Module M1 Section\\
... & ... & ...\\
\bottomrule
\end{tabularx}
\end{table}
\newpage

\maketitle
\newpage

\section{Module M1A (BrowserAction/UserInterface)}
This section consists of the elements that are required for the user interface. It provides the user with a basic format to the display. Since it is a display, it does not have methods and state variables. It just displays the password from the Hasher module. 
\section{Module M1B (BrowserAction/Customization)}
This section consists of css code to customize the user interface and improve the view for the user. It styles all the elements so that they are readable and gives an appealing view.
\section{Module M1C (BrowserAction/EventListeners)}
This section consists of the basic event listeners for the user interface. It has no state variable, only events that can be triggered by actions from the user. It has the following methods:
\newline
\newline
infoView: This method allows the user to click the info button to display more information about the application
\newline
tabChanged: This method detects if the user switches tabs to reflect the current domain
\newline
generate: This is a call to the hasher module to obtain the password and display it.
\newline
togglePass: This method toggles the view of the user key to make it either visible or hidden.

\section{Module M2A (hasher/index)}

\begin{table}[!htbp]
\begin{tabular}{|p{0.5\linewidth} | p{0.5\linewidth}|}
\hline
Name & hasher/index \\
\hline
Imported Indentifiers & $String$ (String data type) \\
& $Object$ (Object data type) \\
& $generateHash$ (hasher/hash)\\
& $applyConstraints$ (hasher/constraints) \\
& $getSalt$ (hasher/salt) \\
\hline
Exported Access Routines & $init$, $generatePassword$ \\
\hline
State Variables & None \\
\hline
State Invariant & $opts \in Object$ \\
\hline
Assumptions & $init$ called first \\
\hline
\end{tabular}
\end{table}

\begin{table}[!htbp]
\caption{Access Routine Semantics}
\begin{tabular}{|p{0.33\linewidth} | p{0.33\linewidth}|p{0.33\linewidth}|}
\hline
Name & IO Relation & Domain \\
\hline
$init()$ & $opts = \{\}$ & None \\
\hline
(1) & $(2)$ & $master \in String \land domain \in String$ \\
\hline
$setOptions(x)$ & $opts = x$ & $x \in Object$ \\
\hline
\end{tabular}
\end{table}

\begin{equation}
generatePassword(master, domain)
\end{equation}

\begin{equation}
result = applyConstraints(generateHash(master+domain+getSalt()), opts)
\end{equation}

\newpage
\section{Module M2B (hasher/hash)}

\begin{table}[!htbp]
\begin{tabular}{|p{0.5\linewidth} | p{0.5\linewidth}|}
\hline
Name & hasher/hash \\
\hline
Imported Indentifiers & $String$ (String data type) \\
& $Buffer$ (Buffer data type) \\
& $hash$ (function that is computationally infeasible to reverse) \\
\hline
Exported Access Routines & $generateHash$ \\
\hline
State Variables & None \\
\hline
State Invariant & $true$ \\
\hline
Assumptions & None \\
\hline
\end{tabular}
\end{table}

\begin{table}[!htbp]
\caption{Access Routine Semantics}
\begin{tabular}{|p{0.33\linewidth} | p{0.33\linewidth}|p{0.33\linewidth}|}
\hline
Name & IO Relation & Domain \\
\hline
$generateHash(x)$ & $result = hash(x) \land result \in String$ & $x \in String \lor x \in Buffer$ \\
\hline
\end{tabular}
\end{table}

\newpage
\section{Module M2C (hasher/unicode)}

\begin{table}[!htbp]
\begin{tabular}{|p{0.5\linewidth} | p{0.5\linewidth}|}
\hline
Name & hasher/unicode \\
\hline
Imported Indentifiers & $String$ (String data type) \\
& $CharacterType$ (Classes of Unicode character (special character, letter, number, alphanum, etc.)) \\
\hline
Exported Access Routines & $fromHexCode$ \\
\hline
State Variables & None \\
\hline
State Invariant & $true$ \\
\hline
Assumptions & None \\
\hline
\end{tabular}
\end{table}

\begin{table}[!htbp]
\caption{Access Routine Semantics}
\begin{tabular}{|p{0.33\linewidth} | p{0.33\linewidth}|p{0.33\linewidth}|}
\hline
Name & IO Relation & Domain \\
\hline
$fromHexCode(code, type)$ & $result \in type \land result \in String$ & $code \in String \land type \in CharacterType$ \\
\hline
\end{tabular}
\end{table}

\newpage
\section{Module M2D (hasher/constraints)}

\begin{table}[!htbp]
\begin{tabular}{|p{0.5\linewidth} | p{0.5\linewidth}|}
\hline
Name & hasher/constraints \\
\hline
Imported Indentifiers & $String$ (String data type) \\
& $Object$ (Object data type) \\
& $CharacterType$ (Classes of Unicode character (special character, letter, number, alphanum, etc.)) \\
\hline
Exported Access Routines & $applyConstraints$ \\
\hline
State Variables & None \\
\hline
State Invariant & $true$ \\
\hline
Assumptions & None \\
\hline
\end{tabular}
\end{table}

\begin{table}[!htbp]
\caption{Access Routine Semantics}
\begin{tabular}{|p{0.33\linewidth} | p{0.33\linewidth}|p{0.33\linewidth}|}
\hline
Name & IO Relation & Domain \\
\hline
$applyConstraints(x, opts)$ & (3) & $x \in String \land opts \in Object$ \\
\hline
\end{tabular}
\end{table}

\begin{equation}
result \in String \land ((	\exists_{n} type \in result)\forall type \in keys(opts) \land n = opts[type])
\end{equation}

\newpage
\section{Module M2E (hasher/salt)}

\begin{table}[!htbp]
\begin{tabular}{|p{0.5\linewidth} | p{0.5\linewidth}|}
\hline
Name & hasher/salt \\
\hline
Imported Indentifiers & $String$ (String data type) \\
& $Buffer$ (String data type) \\
\hline
Exported Access Routines & $setSalt$, $getSalt$ \\
\hline
State Variables & $salt$ \\
\hline
State Invariant & $salt \in Buffer \lor salt = null$ \\
\hline
Assumptions & None \\
\hline
\end{tabular}
\end{table}

\begin{table}[!htbp]
\caption{Access Routine Semantics}
\begin{tabular}{|p{0.33\linewidth} | p{0.33\linewidth}|p{0.33\linewidth}|}
\hline
Name & IO Relation & Domain \\
\hline
$setSalt(x)$ & $salt = x$ & $x \in Buffer$ \\
\hline
$getSalt()$ & $result = toString(salt)$ & $true$ \\
\hline
\end{tabular}
\end{table}

\newpage
\section{Module M3A (options/constraints)}

\begin{table}[!htbp]
\begin{tabular}{|p{0.5\linewidth} | p{0.5\linewidth}|}
\hline
Name & options/constraints \\
\hline
Imported Indentifiers & $\N$ Natural numbers $(0, \infty]$ \\
\hline
Exported Access Routines & $setLength$, $getLength$ \\
\hline
State Variables & $length$ \\
\hline
State Invariant & $length \in \N$ \\
\hline
Assumptions & None \\
\hline
\end{tabular}
\end{table}

\begin{table}[!htbp]
\caption{Access Routine Semantics}
\begin{tabular}{|p{0.33\linewidth} | p{0.33\linewidth}|p{0.33\linewidth}|}
\hline
Name & IO Relation & Domain \\
\hline
$setLength(x)$ & $length = x$ & $x \in \N$ \\
\hline
$getLength()$ & $result = length$ & $true$ \\
\hline
\end{tabular}
\end{table}


\newpage
\section{Module M3B (options/display)}

\begin{table}[!htbp]
\begin{tabular}{|p{0.5\linewidth} | p{0.5\linewidth}|}
\hline
Name & options/display \\
\hline
Imported Indentifiers & $Object$ (Object data type) \\
\hline
Exported Access Routines & $setOptions$, $getOptions$ \\
\hline
State Variables & $opts$ \\
\hline
State Invariant & $opts \in Object$ \\
\hline
Assumptions & None \\
\hline
\end{tabular}
\end{table}

\begin{table}[!htbp]
\caption{Access Routine Semantics}
\begin{tabular}{|p{0.33\linewidth} | p{0.33\linewidth}|p{0.33\linewidth}|}
\hline
Name & IO Relation & Domain \\
\hline
$setOptions(x)$ & $opts= x$ & $x \in Object$ \\
\hline
$getOptions()$ & $result = opts$ & $true$ \\
\hline
\end{tabular}
\end{table}


\end{document}
